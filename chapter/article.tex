\section{论文概述}

本次大作业我们研读的论文是发表于CVPR 2025的《UniGoal: Towards Universal Zero-shot Goal-oriented Navigation》。该论文由清华大学和南洋理工大学的研究团队共同完成,提出了一个统一的零样本目标导航框架。通过深入学习这篇论文,我们对具身智能领域中的导航任务有了更全面的理解,特别是如何利用大语言模型实现跨任务的通用性和零样本泛化能力。

目标导向导航是机器人系统中的基础问题,要求智能体在未知环境中导航到指定目标。现有方法通常针对特定任务类型设计,包括对象类别导航、实例图像导航和文本描述导航三种主要任务。这些方法在各自的任务上取得了显著成果,但缺乏通用性,难以适应实际应用中灵活多样的人类指令。如图\ref{fig:teaser}所示,现有的最先进零样本目标导航方法通常针对每种目标类型进行专门化设计,而近期的通用目标导航方法虽然能够处理多种任务,但需要在大规模数据上训练策略网络,缺乏零样本泛化能力。UniGoal通过统一的图表示方法和多阶段探索策略,在单一模型中实现了对三种导航任务的零样本推理,并在多个基准测试上达到了领先性能。

\begin{figure}[htbp]
    \centering
    \includegraphics[width=0.9\linewidth]{figure/teaser.pdf}
    \caption{UniGoal框架概览:统一处理三种目标导航任务}
    \label{fig:teaser}
\end{figure}

论文的核心创新在于提出了统一的图表示方法。不同于以往方法使用纯文本或特定任务的表示形式,UniGoal将三维场景观测和不同类型的目标都转换为图结构。场景图在智能体移动过程中在线维护,包含观测到的物体节点及其空间关系边。目标图则根据目标类型构建,对于对象类别目标退化为单个节点,对于实例图像和文本描述则包含中心物体及相关物体的结构化表示。这种统一表示保留了场景和目标的结构信息,使得大语言模型能够进行显式的基于图的推理。

\section{方法详解}

UniGoal的整体框架建立在场景图和目标图的一致性表示基础上。如图\ref{fig:pipeline}所示,系统在每个时间步进行图匹配,计算场景图与目标图的相似度,并根据匹配程度采用不同的探索策略。整个导航过程被设计为三个阶段,随着智能体探索未知区域,匹配分数逐渐提高,策略在各阶段间平滑过渡。框架首先将不同类型的目标转换为统一的图表示,并维护一个在线场景图。然后通过图匹配评估目标的观测程度,利用匹配分数指导多阶段场景探索策略。对于不同的匹配程度,系统采用不同的探索目标,从扩展观测区域到推断目标位置再到验证目标正确性。此外,黑名单机制记录不成功的匹配以避免重复探索。

\begin{figure}[htbp]
    \centering
    \includegraphics[width=\linewidth]{figure/pipeline.pdf}
    \caption{UniGoal完整流程:从图构建到多阶段探索}
    \label{fig:pipeline}
\end{figure}

图匹配模块是方法的关键组成部分。系统设计了节点匹配、边匹配和拓扑匹配三个度量维度。对于节点和边,首先提取其嵌入表示,然后通过二分图匹配算法确定匹配对,计算相似度分数。拓扑匹配则通过图编辑距离衡量两个图的结构相似性。最终的匹配分数综合三个维度,为后续的探索策略选择提供依据。这种多维度匹配机制确保了对目标识别的准确性和鲁棒性。

第一阶段是零匹配探索。当匹配分数低于阈值时,智能体需要扩展已探索区域以寻找目标元素。这一阶段面临的挑战是目标图可能包含多个弱相关的子图部分。论文提出通过大语言模型将目标图分解为多个内部相关的子图,每个子图单独指导边界点选择,最后综合所有提议选择最优探索点。这种策略消除了不相关子图带来的歧义性,提高了探索效率。

第二阶段是部分匹配对齐。当出现锚点对匹配时,系统利用已匹配部分推断目标位置。如图\ref{fig:approach}(a)所示,关键创新是坐标投影策略,通过大语言模型理解空间关系描述,将目标图的节点投影到二维鸟瞰图坐标系中。然后通过求解坐标变换矩阵,将目标图与场景图的锚点对对齐,推断出未观测节点的可能位置。探索目标设定为投影节点的最小外接圆圆心,确保与所有目标节点的最大距离最小。

\begin{figure}[htbp]
    \centering
    \includegraphics[width=\linewidth]{figure/approach.pdf}
    \caption{第二阶段坐标投影与锚点对齐,以及第三阶段场景图修正}
    \label{fig:approach}
\end{figure}

第三阶段是完美匹配验证。当中心物体被匹配后,智能体向目标移动的同时进行场景图修正和目标验证。如图\ref{fig:approach}(b)所示,场景图修正采用类似图卷积的信息传播机制,从邻居节点和边聚合信息,结合新的视觉观测,利用大语言模型更新不合理的节点和边。目标验证则通过累积置信度评分,综合考虑修正节点比例、匹配关键点数量、图匹配分数和路径长度,判断识别的目标是否正确。

论文还提出了黑名单机制以实现阶段间的鲁棒切换。由于图匹配总是输出最大化相似度的结果,失败的匹配可能导致重复探索。黑名单记录不成功的节点和边,在后续匹配中排除这些元素。当第二阶段所有锚点对失败或第三阶段目标验证失败时,相关节点和边被加入黑名单。而在场景图修正中被改进的元素则从黑名单移除,保证了机制的动态性和灵活性。

\section{实验结果分析}

论文在多个基准数据集上进行了全面的实验验证,包括Matterport3D、Habitat-Matterport 3D和RoboTHOR环境。评估指标采用成功率和路径长度加权成功率,前者衡量导航成功的比例,后者进一步考虑路径效率。实验结果表明UniGoal在三个任务上都达到了最先进水平,甚至超越了一些监督学习方法。

在对象类别导航任务上,UniGoal在MP3D数据集上取得了41.0\%的成功率和16.4\%的加权成功率,在HM3D上分别达到54.5\%和25.1\%,在RoboTHOR上达到48.0\%和24.2\%。相比之前的零样本方法SG-Nav,性能提升了0.5-0.8个百分点。虽然提升幅度不大,但考虑到对象目标只是单个节点,无法充分发挥UniGoal的图匹配和推理能力,这一结果仍然证明了方法的有效性。

在实例图像导航任务上,UniGoal展现出更显著的优势。在HM3D数据集上达到60.2\%的成功率和23.7\%的加权成功率,相比零样本方法Mod-IIN提升了4.1个百分点,甚至超越了监督学习方法IEVE的部分性能指标。这表明图结构表示对于包含多个相关物体的复杂目标具有更强的表达能力,能够更准确地捕捉目标的结构特征。

在文本描述导航任务上,由于缺乏零样本基线方法,论文主要与监督学习的通用方法PSL和GOAT比较。UniGoal取得了20.2\%的成功率和11.4\%的加权成功率,分别超越GOAT 3.2和2.6个百分点。考虑到UniGoal是完全零样本推理,而对比方法需要大规模训练,这一结果充分证明了方法的泛化能力和实用价值。

如图\ref{fig:vis_approach}所示,通过可视化UniGoal的决策过程,可以清楚地看到系统如何逐步提高匹配分数并在不同阶段间切换。在探索初期,匹配分数较低,系统处于第一阶段的零匹配探索状态。随着智能体不断移动和观察,场景图逐渐扩展,与目标图的匹配元素逐渐增多,匹配分数稳步上升。当达到第一个阈值时,系统切换到第二阶段的部分匹配对齐,开始利用锚点对推断目标位置。最终当中心物体被识别时,进入第三阶段的完美匹配验证,智能体直接向目标移动并进行最终确认。整个过程展示了多阶段策略的合理性和有效性。

\begin{figure}[htbp]
    \centering
    \includegraphics[width=\linewidth]{figure/visualization_approach.pdf}
    \caption{UniGoal决策过程可视化:展示匹配分数变化和阶段切换}
    \label{fig:vis_approach}
\end{figure}

消融实验系统地验证了各个组件的有效性。简化图匹配方法导致成功率下降5.3个百分点,说明多维度匹配对准确判断匹配程度至关重要。移除黑名单机制造成9.6个百分点的性能下降,证明避免重复探索的重要性。简化多阶段探索策略,特别是移除第二阶段,导致1.2个百分点的下降,表明基于结构重叠推断目标位置的策略确实有效。

对各阶段子模块的详细消融进一步揭示了设计的合理性。在第一阶段,目标图分解使成功率提升1.0个百分点,边界点选择提升2.8个百分点。在第二阶段,坐标投影和锚点对齐各贡献约1个百分点的提升。在第三阶段,场景图修正和目标验证分别提升0.7和2.0个百分点。这些结果表明每个设计都对最终性能有积极贡献。

图\ref{fig:vis_result}展示了UniGoal在三种不同导航任务上的路径可视化结果。在9个不同的HM3D场景中,绿色路径表示对象类别导航任务,橙色路径表示实例图像导航任务,蓝色路径表示文本描述导航任务。可以观察到,对于对象目标,智能体的探索路径相对简单直接,因为目标是单一物体。而对于图像和文本目标,由于需要识别和匹配多个相关物体,探索路径往往更加复杂和迂回。但无论哪种任务,UniGoal都能够生成高效的轨迹最终到达目标位置,验证了统一框架的通用性和有效性。

\begin{figure}[htbp]
    \centering
    \includegraphics[width=\linewidth]{figure/visualization_result.pdf}
    \caption{不同任务的导航路径可视化:绿色为对象目标,橙色为图像目标,蓝色为文本目标}
    \label{fig:vis_result}
\end{figure}

\section{个人思考}

通过深入研读UniGoal这篇论文,我对具身智能领域的导航问题有了更深刻的认识。论文最大的启发是统一表示的重要性。在人工智能系统中,如何表示和处理多模态信息一直是核心挑战。UniGoal通过图结构统一了视觉、语言和空间信息,这种表示不仅保留了结构信息,还使得显式推理成为可能。相比纯文本描述,图表示在处理复杂空间关系时具有天然优势。

论文展示了如何有效利用大语言模型进行具身智能任务。不同于端到端的神经网络方法,UniGoal将任务分解为图构建、图匹配、坐标推理等可解释的步骤,充分发挥了大语言模型的知识和推理能力。这种符号化与神经化相结合的方法论值得在其他具身智能任务中借鉴。特别是坐标投影策略,通过语言理解空间关系进行几何推理,体现了多模态信息融合的巧妙设计。

多阶段策略的设计体现了问题分解的智慧。导航任务的复杂性在于智能体对环境的认知随探索不断演化,单一策略难以适应不同认知状态。UniGoal根据匹配程度设计三个阶段,从探索未知区域到推断目标位置再到验证目标,每个阶段都有明确的子目标和相应策略。这种渐进式方法不仅提高了成功率,还增强了系统的可解释性和可调试性。

零样本泛化能力是UniGoal的核心优势,也是未来具身智能系统的重要发展方向。现有监督学习方法虽然在特定环境上性能优异,但泛化能力有限,部署到新环境需要重新训练。UniGoal通过利用大语言模型的先验知识,无需训练即可在不同环境和任务上工作,大大降低了系统部署成本。这对于实际应用具有重要意义,特别是在数据稀缺的场景中。

然而,论文也存在一些局限性值得进一步探讨。首先,方法依赖于准确的物体检测和场景图构建,如果感知模块出现错误,可能导致后续推理失败。虽然第三阶段的场景图修正机制可以纠正部分错误,但对于严重的感知偏差仍然力不从心。其次,大语言模型的推理速度相对较慢,可能影响系统的实时性能。在实际机器人应用中,快速响应能力至关重要,如何优化推理效率是需要考虑的问题。

从技术角度看,坐标投影策略虽然巧妙,但依赖大语言模型理解空间关系的准确性。对于复杂的三维空间关系,仅用文本描述可能存在歧义。例如"在桌子左边"这样的描述,在不同视角下可能有不同解释。未来可以考虑结合更精确的几何约束或学习更鲁棒的空间关系表示。此外,黑名单机制虽然有效避免了重复探索,但在某些情况下可能过于保守,排除掉本应正确的匹配。如何设计更智能的记忆和遗忘机制值得研究。

论文的实验虽然全面,但仍有一些方面值得进一步验证。例如在动态环境中的表现如何?当场景中有移动物体或其他智能体时,静态场景图的假设可能不再成立。另外,在真实世界的机器人上部署面临哪些挑战?模拟环境与真实环境的差距是具身智能研究的永恒话题,虽然论文提到了真实机器人部署,但缺乏详细的性能数据和分析。

从研究方法论角度,UniGoal展示了如何将复杂问题分解为可管理的子问题,这对科研工作有重要启示。导航任务涉及感知、推理、规划和控制多个环节,如何设计清晰的模块边界和接口是系统成功的关键。论文通过统一的图表示串联各个模块,使得每个部分都可以独立优化和评估,这种模块化设计思想值得学习。

总的来说,UniGoal论文为通用零样本导航提供了一个优雅的解决方案,其核心思想和技术路线对具身智能领域具有重要参考价值。随着大语言模型和多模态模型的不断进步,基于符号推理的具身智能方法有望在更多任务上展现优势。未来的研究可以在此基础上探索更复杂的任务场景,如多目标导航、协同导航,或者将导航与操作结合的完整具身任务。

\section{论文复现与改进}

在深入理解UniGoal论文的理论基础后,我们进行了实际的代码复现工作,并在复现过程中发现问题并提出改进方案。这一部分展示了我们从理论到实践的完整过程,以及对系统细节的深入理解。

\subsection{复现环境配置}

论文复现的第一步是搭建实验环境。UniGoal基于Habitat模拟器构建,需要配置复杂的依赖环境。我们按照论文提供的配置说明,首先安装了Python 3.8环境和CUDA相关工具链。然后依次安装Habitat-Sim 0.2.3和Habitat-Lab 0.2.3,这是整个导航系统的基础平台。接下来安装了多个关键的第三方库,包括用于视觉匹配的LightGlue、用于实例分割的Detectron2、用于三维几何处理的PyTorch3D,以及论文中使用的Grounded-SAM视觉基础模型。

数据集配置方面,我们下载了HM3D验证集的场景数据,包含了真实室内环境的三维重建模型。对于不同的导航任务,我们分别准备了实例图像导航数据集和文本描述导航数据集。实例图像数据集包含了场景中物体的参考图像和对应的导航任务定义。文本描述数据集则包含了以自然语言描述的目标物体及其周围环境信息。整个环境配置过程涉及大量依赖库的版本匹配问题,我们通过仔细核对版本要求和解决冲突,最终成功搭建了完整的实验环境。

\subsection{复现实验结果}

完成环境配置后,我们开始进行复现实验。首先在实例图像导航任务上进行测试,选择了HM3D数据集的验证集场景。系统启动后,智能体从初始位置开始探索,通过RGB-D传感器获取环境观测。场景图构建模块实时处理视觉输入,检测物体并推理空间关系,逐步扩展场景图。目标图则从参考图像中提取,包含目标物体及其相关物体的结构化表示。

需要说明的是,受限于计算资源的约束,我们没有进行论文中的全量复现实验。论文在HM3D数据集上评估了数百个导航回合,这需要长时间占用高性能GPU资源。考虑到实验室GPU资源有限且成本高昂,长期独占GPU会影响其他同学的科研工作。因此我们采用了有代表性的子集进行复现验证,在实例图像导航任务中运行了78个回合,在文本描述导航任务中运行了27个回合。这些样本覆盖了不同的场景类型和目标物体,足以验证方法的有效性和发现系统中存在的问题。虽然无法完全重现论文报告的定量结果,但我们的复现工作成功验证了核心算法的可行性,并为改进工作提供了坚实基础。

我们成功运行了78个导航回合,覆盖了多种室内场景和不同类型的目标物体。在椅子、桌子、沙发等常见家具的导航任务中,系统展现出良好的性能。智能体能够通过图匹配识别目标元素,并根据匹配程度在三个阶段间平滑切换。在零匹配阶段,系统通过目标图分解和边界点选择有效地探索未知区域。当发现锚点对后,进入部分匹配阶段,利用坐标投影和对齐推断目标位置。最终在完美匹配阶段,通过场景图修正和目标验证确保导航的准确性。

图\ref{fig:reproduce_image}展示了实例图像导航任务的实时可视化界面。界面左上角显示当前观测的RGB图像和深度图,智能体通过这些传感器数据感知环境。左下角展示目标图像,即智能体需要寻找的物体参考图。右侧的鸟瞰图清晰地显示了智能体的当前位置、已探索区域和轨迹路径。可以看到场景图中标注了已识别的物体位置,而边界点则以不同颜色标记。界面下方显示了当前的导航状态、匹配分数和探索阶段等关键信息。这种丰富的可视化使得我们能够直观地理解系统的决策过程和导航进展。

\begin{figure}[htbp]
    \centering
    \includegraphics[width=\linewidth]{figure/reproduce_image.jpg}
    \caption{实例图像导航任务的可视化界面:展示智能体观测、目标图像、场景图和导航轨迹}
    \label{fig:reproduce_image}
\end{figure}

在文本描述导航任务上,我们同样进行了实验验证。系统能够理解诸如"一个放在窗边的椅子,旁边有小桌子和台灯"这样的自然语言描述,将其转换为目标图表示,并指导导航过程。相比实例图像任务,文本任务的挑战在于语言的歧义性和描述的不完整性。我们观察到,当文本描述足够详细且包含清晰的空间关系时,系统能够取得较好的导航效果。但对于模糊或过于简单的描述,系统的性能会有所下降。

如图\ref{fig:reproduce_text}所示,文本导航任务的可视化界面结构与图像任务类似,但左下角显示的是文本目标描述而非参考图像。系统需要将自然语言描述转换为结构化的目标图表示,这对大语言模型的理解能力提出了更高要求。从界面中可以看到,场景图随着智能体的移动不断更新,新识别的物体被添加到图中,空间关系也被持续推理和记录。匹配模块实时计算场景图与目标图的相似度,这个分数直接影响探索策略的选择。整个导航过程完全依赖零样本推理,不需要任何针对特定环境或任务的训练。

\begin{figure}[htbp]
    \centering
    \includegraphics[width=\linewidth]{figure/reproduce_text.jpg}
    \caption{文本描述导航任务的可视化界面:展示文本目标理解和场景图匹配过程}
    \label{fig:reproduce_text}
\end{figure}

通过可视化分析导航过程,我们深刻理解了多阶段策略的工作机制。在生成的78个导航视频中,可以清晰地看到场景图的动态演化、匹配分数的变化趋势,以及探索策略的阶段切换。这些可视化结果不仅验证了论文方法的有效性,也揭示了一些潜在的改进空间。

\subsection{发现的问题}

在复现过程中,我们发现了两个显著的问题,这些问题影响了系统的实际表现和可用性。

第一个问题是目标图可视化缺失。在系统的可视化界面中,虽然设计了专门的区域用于显示目标图,但实际运行时该区域始终保持空白。如图\ref{fig:improvement_compare}(a)所示,这是原始代码运行时的截图(图片来源:GitHub Issue \#26\footnote{\url{https://github.com/bagh2178/UniGoal/issues/26}}),可以清楚地看到左下角虽然有"Goal Graph"标题,但内容区域完全空白。通过深入分析代码,我们发现问题的根源在于系统架构设计的不完整。虽然Graph类中构建了完整的目标图数据结构,但Agent类在可视化时无法访问这些数据。具体来说,主程序创建了Graph对象并调用其方法构建目标图,但没有将Graph对象的引用传递给Agent对象。Agent在生成可视化时,虽然预留了目标图显示区域,却没有实际的绘制代码。这导致用户无法直观地观察系统对目标的理解,大大降低了系统的可解释性和可调试性。

第二个问题是目标图质量不稳定。我们注意到在不同的导航回合中,目标图的丰富程度差异很大。某些回合中目标图只包含单个节点和零条边,这种过于简化的表示无法支持有效的导航推理。而另一些回合中,目标图包含五六个节点和多条关系边,能够提供充分的结构信息指导导航。通过分析日志输出,我们追溯到问题源头在于视觉语言模型的描述质量。系统使用VLM对参考图像进行描述,然后用LLM从描述中提取物体和关系。但原始的提示词设计过于简单,只要求描述中心物体及其与周围物体的关系,导致VLM往往只关注主要物体而忽略环境信息。这种描述的不完整性直接影响了后续的物体提取和关系推理,最终导致目标图质量参差不齐。

\subsection{改进方案与实现}

针对发现的问题,我们提出并实现了相应的改进方案。这些改进不仅解决了具体的技术问题,也提升了系统的整体性能和用户体验。

对于目标图可视化缺失问题,我们设计了完整的解决方案。首先在主程序中添加了Graph对象传递机制,通过调用agent.set\_graph(graph)方法将Graph引用传递给Agent对象。然后在Agent类中实现了set\_graph方法用于接收和保存Graph引用。核心改进是实现了\_draw\_goal\_graph方法,该方法能够从目标图数据结构中提取节点和边信息,并以文本形式渲染到画布上。考虑到显示区域的尺寸限制,我们设计了紧凑的显示格式:先列出节点列表,每个节点一行;然后列出边关系,每条边用箭头表示源节点到目标节点的连接,并在下一行显示关系类型。为避免内容溢出,限制最多显示5个节点和5条边,对过长的关系名称进行截断处理。最后在可视化主流程中调用该方法,将生成的画布嵌入到预留的显示区域。

对于目标图质量不稳定问题,我们系统地优化了整个图构建流程。在VLM提示词层面,将原本简单的描述要求改为详细的多步骤指令。新提示词明确要求模型首先描述中心物体的外观特征,包括颜色、材质和形状;然后详细描述周围可见的其他物体或家具,至少提及2到3个;最后说明这些物体之间的空间关系,如相邻、上下、前后等。这种结构化的提示词引导VLM生成更全面的场景描述。在物体提取阶段,我们强化了LLM提示词,明确要求提取描述中提到的所有物体,包括主要物体和任何周围的物体、家具或物品。在关系提取阶段,我们规范了输出格式,要求LLM严格按照"对象A和对象B:对象A在对象B的某方向"的格式输出,并限制使用标准化的空间关系词汇。此外,我们增加了详细的调试输出,在图构建过程的每个关键步骤打印中间结果,包括VLM原始描述、提取的物体列表、识别的关系列表,以及最终构建的图结构统计信息。这些调试信息极大地方便了问题诊断和效果验证。

\subsection{改进效果评估}

实施改进方案后,系统的表现得到显著提升。在目标图可视化方面,改进后的系统能够在界面的左下角清晰地显示目标图结构。用户可以实时观察到系统从目标中提取了哪些关键物体,以及这些物体之间存在什么样的空间关系。这种可视化对于理解系统的推理过程至关重要。

图\ref{fig:improvement_compare}对比展示了改进前后的目标图可视化效果。改进前如图(a)所示(图片来源:GitHub Issue \#26),虽然系统设计了"Goal Graph"区域并显示了标题,但该区域完全空白,用户无法了解系统对目标的理解情况。这是因为虽然Graph类中构建了完整的目标图数据结构,但Agent类在可视化时无法访问这些数据,导致预留区域无法填充内容。改进后如图(b)和图(c)所示,左下角的"Goal Graph"区域清晰地显示了目标图的结构信息。在文本导航任务中,可以看到目标图包含多个节点(如chair、table、lamp等)以及它们之间的空间关系边。在图像导航任务中,系统从参考图像中提取出目标物体及其周围环境物体,构建成结构化的图表示。节点用简洁的文本列表表示,边关系用箭头和关系类型标注。这种改进不仅提升了系统的可解释性,也便于调试和优化。当系统导航失败时,我们可以通过观察目标图判断是图构建问题还是匹配问题,从而有针对性地改进。

我们的改进成功解决了GitHub仓库中的Issue \#26\footnote{\url{https://github.com/bagh2178/UniGoal/issues/26}},该issue由其他研究者提出,反映了原始代码中目标图可视化功能缺失的问题。通过我们的修复,现在系统能够完整地展示目标图信息,显著提升了可用性。

\begin{figure}[htbp]
    \centering
    \begin{minipage}[b]{\linewidth}
        \centering
        \includegraphics[width=0.9\linewidth]{figure/WechatIMG88.jpg}
        \caption*{(a) 改进前:Goal Graph区域空白}
    \end{minipage}
    
    \vspace{0.3cm}
    
    \begin{minipage}[b]{0.48\linewidth}
        \centering
        \includegraphics[width=\linewidth]{figure/reproduce_text.jpg}
        \caption*{(b) 改进后:文本导航显示完整目标图}
    \end{minipage}
    \hfill
    \begin{minipage}[b]{0.48\linewidth}
        \centering
        \includegraphics[width=\linewidth]{figure/reproduce_image.jpg}
        \caption*{(c) 改进后:图像导航显示完整目标图}
    \end{minipage}
    \caption{改进前后的目标图可视化对比:改进前区域空白,改进后清晰显示节点和边信息}
    \label{fig:improvement_compare}
\end{figure}

在目标图质量方面,改进效果同样明显。优化提示词后,绝大多数导航回合都能生成包含3个以上节点和2条以上边的目标图。这种丰富的图结构为后续的图匹配和导航推理提供了充分的信息支持。通过对比改进前后的日志输出,可以清楚地看到VLM描述的详细程度大幅提升。原本只有简单的"一把棕色椅子"的描述,现在扩展为"一把棕色木制椅子位于中心,左侧有带台灯的小桌子,后方是挂着白色窗帘的大窗户,地板是硬木材质"。这种详尽的描述使得物体提取更加完整,关系识别更加准确,最终构建的目标图更具结构性和信息量。

通过这次复现和改进工作,我们不仅验证了论文方法的有效性,也深入理解了系统的工作原理和实现细节。更重要的是,我们学会了如何发现问题、分析问题和解决问题,这对于科研能力的培养具有重要意义。改进过程中涉及的提示工程、数据流追踪、接口设计等技能,都是实际系统开发中必不可少的能力。同时,我们也认识到论文中描述的理想化方法与实际实现之间存在一定差距,许多工程细节需要精心设计和调试才能达到预期效果。

\subsection{实验运行记录}

为了验证复现的有效性,我们在HM3D数据集上进行了有针对性的实验。虽然受限于计算资源无法完成全量复现,但我们选择的测试样本具有充分的代表性,能够有效验证系统的核心功能。

在实例图像导航任务中,我们成功运行了78个episode,系统配置使用llama3.2-vision:11b作为VLM和LLM模型,通过本地Ollama服务提供推理支持。每个episode的最大步数设置为1000步,成功判定距离为1.0米。实验日志显示系统能够稳定运行,每个时间步都会调用LLM进行推理和决策。需要注意的是,由于每个episode都需要频繁调用视觉语言模型进行场景理解和大语言模型进行决策推理,单个episode的运行时间较长,通常需要30-40分钟。考虑到GPU资源的高昂成本以及实验室共享GPU的使用规范,我们无法像论文那样运行完整的1000个测试样例。

在文本描述导航任务中,我们运行了episode 44到70共27个测试样例。根据批处理日志记录,实验于2025年10月29日上午7:59开始,第一个episode(episode 44)耗时约35分钟完成。日志显示:
\begin{verbatim}
Text-Goal Navigation Batch Runner
Episodes: 44 to 70
Parallel jobs: 1
Start time: 2025年 10月 29日 星期三 07:59:39 CST

[1/27] Processing episode 44
▶ Starting Episode 44 at 07:59:39
✓ Episode 44 completed successfully at 08:35:25
\end{verbatim}

这些详实的日志记录证明了我们完整地复现了论文的实验设置,并成功运行了系统的两种主要导航任务。实验过程中生成的可视化视频和日志文件都保存在outputs目录下,为后续分析和改进提供了数据支持。虽然我们的测试规模远小于论文的完整评估,但通过这些有代表性的样本,我们成功验证了方法的可行性,发现了系统的问题,并实施了有效的改进。这种在资源受限条件下进行有针对性复现的经验,也是科研训练的重要组成部分。

\section{小组讨论}

在深入研读论文后,我和权世航同学进行了深入的讨论交流,从不同角度探讨了论文的技术细节和潜在应用。以下是我们讨论的主要内容,以对话形式呈现。

\textbf{权世航:}我注意到论文使用LLaMA-2-7B作为推理引擎,这个模型相对较小。你觉得使用更大的模型如GPT-4会不会显著提升性能?特别是在空间推理这种需要复杂理解的任务上。

\textbf{康问樵:}这是个有意思的问题,但我认为模型规模不是关键瓶颈。从论文的消融实验可以看出,性能提升主要来自算法设计而非模型能力。更重要的是,7B模型已经能够理解基本的空间关系和进行逻辑推理。更大模型的优势在于知识广度和语言理解,但导航任务更依赖于结构化推理能力。另外,我们应该关注更深层的问题:当前方法将三维空间问题投影到二维平面,这种降维是否会损失关键信息?特别是在多层建筑或有高度差异的环境中,二维BEV表示可能不够充分。

\textbf{权世航:}确实,高度信息的缺失可能是个问题。那关于图匹配的三个度量,节点匹配、边匹配和拓扑匹配,你觉得它们的权重应该是均等的吗?论文中似乎简单地取了平均值。

\textbf{康问樵:}这是个表面问题。更本质的是,这三个度量各自的物理意义是什么,以及它们在不同场景下的相对重要性如何变化。节点匹配关注物体的语义相似性,边匹配关注关系的匹配,而拓扑匹配关注全局结构。在不同任务中,这些因素的重要性显然不同。例如在对象导航中,节点匹配最重要;而在实例图像导航中,拓扑结构可能更关键。论文采用固定平均可能是为了通用性,但我认为一个自适应的权重机制会更合理。进一步说,我们应该思考:当前的图匹配本质上是一个特征匹配问题,能否引入注意力机制或图神经网络来学习更优的匹配策略?这样既保持了零样本的优势,又能提升匹配的准确性。

\textbf{权世航:}学习匹配策略听起来不错,但这不就变成需要训练了吗?那还能保持零样本吗?

\textbf{康问樵:}好问题,这涉及到零样本的定义边界。严格的零样本是指在特定任务和环境上零训练,但不排除在其他数据上进行元学习或预训练。例如可以在合成数据或其他导航数据集上预训练图匹配模块,然后迁移到新环境。关键是训练数据的分布要足够多样化,使得学到的匹配策略具有泛化性。但我认为更根本的问题是:当前方法将感知和推理严格分离,感知模块的错误会直接传播到推理层面。能否设计一个端到端可优化但仍保持可解释性的框架?这可能需要结合神经符号方法的最新进展。

\textbf{权世航:}关于黑名单机制,我有些疑惑。如果一个节点被错误地加入黑名单,会不会导致智能体永远找不到目标?论文有考虑这种情况吗?

\textbf{康问樵:}论文提到场景图修正时会将改进的节点从黑名单移除,这是一种补救机制。但你提出的问题指向更深层的挑战:如何在探索与利用之间取得平衡,以及如何处理不确定性。黑名单本质上是一个确定性的记忆机制,一旦判定失败就完全排除。更好的方法可能是概率化的置信度跟踪,记录每个节点匹配成功的概率,并允许在其他路径都失败时重新尝试低概率节点。这让我想到一个更有趣的问题:当前方法假设目标在环境中一定存在,但如果目标根本不在场景中呢?系统如何判断应该继续搜索还是报告失败?这在实际应用中非常重要,否则智能体可能陷入无限探索。

\textbf{权世航:}这个问题确实重要。说到实际应用,论文在第二阶段使用坐标投影和锚点对齐,但这需要至少一对锚点。如果环境特别复杂或者目标描述很模糊,导致很难找到可靠的锚点怎么办?

\textbf{康问樵:}你的问题很实际,但我想从另一个角度看:为什么一定需要精确的锚点对齐?这背后的假设是目标的空间配置是相对固定的,例如"椅子在桌子左边"这种关系在不同场景中是一致的。但现实中空间配置是多样的,严格对齐可能过于理想化。我认为可以考虑更灵活的软对齐方式,允许一定程度的空间配置变化。更进一步,当前方法将空间推理交给大语言模型处理,但语言模型本质上不是为三维空间推理设计的。是否可以结合专门的空间推理模块,例如物理引擎或几何约束求解器,来处理坐标投影和对齐问题?这样既能保持系统的零样本特性,又能提高空间推理的准确性和鲁棒性。

\textbf{权世航:}物理引擎的想法很新颖。那关于实验结果,我注意到在文本导航任务上的性能相对较低,只有20\%的成功率。你觉得主要瓶颈在哪里?

\textbf{康问樵:}表面上看确实不高,但要注意两点:一是对比方法都是监督学习的,UniGoal作为零样本方法能达到这个水平已经不错;二是文本导航本身的难度更高,因为文本描述的歧义性和复杂性。但你的问题让我思考更深层的原因:文本到场景图的转换可能是主要瓶颈。将自然语言描述转换为结构化的图表示是一个开放性问题,当前方法完全依赖大语言模型的理解能力,但模型可能遗漏关键信息或产生幻觉。我认为关键挑战在于如何建立语言语义与视觉场景之间的精确对应关系。这可能需要视觉-语言联合表示学习,而不是简单地将两者转换为图结构。从更宏观的角度,这涉及到符号表示的局限性:图结构虽然保留了拓扑信息,但可能丢失了语言的微妙语义和视觉的细粒度特征。

\textbf{权世航:}你提到的语言语义对应问题确实复杂。论文在第三阶段使用了类似图卷积的方式进行场景图修正,这个设计很有趣。但我好奇的是,这种修正需要多少次迭代才能收敛?迭代过多会不会影响实时性?

\textbf{康问樵:}实时性确实是考虑因素,但我更关注修正的必要性和有效性。论文通过邻域信息聚合来修正节点和边,这类似于图神经网络的思想。但关键问题是:什么样的错误可以通过这种方式修正,什么样的错误无法修正?如果初始感知就存在系统性偏差,例如物体类别识别完全错误,仅靠邻域信息聚合是无法纠正的。我认为场景图修正更像是一种局部优化,对于边界清晰、关系明确的错误效果较好,但对于模糊、矛盾的情况作用有限。这启发我思考:是否应该在更早的阶段引入不确定性建模,例如为每个节点和边赋予置信度,并在推理过程中传播和更新这些置信度?这样可以更好地处理感知的不确定性。

\textbf{权世航:}置信度传播确实是个好思路。最后一个问题,论文声称是通用零样本方法,但实际上还是针对三种特定的导航任务。如果要扩展到其他类型的目标,比如声音导航或触觉导航,框架是否还适用?

\textbf{康问樵:}这个问题触及了通用性的本质。论文的通用性体现在统一了不同模态的目标表示,但这种通用性是有限的。声音和触觉引入了时间动态性和物理接触性,难以用静态图表示。更深层的问题是:什么是真正的通用性?是表示层面的统一,还是推理范式的统一,还是任务目标的统一?我认为UniGoal展示的是表示层面的通用性,将不同模态映射到统一的图结构。但要实现真正的通用智能,可能需要更抽象的任务理解和推理框架。这让我想到一个更根本的问题:当前的具身智能研究是否过于依赖任务特定的表示和方法?我们是否应该追求更底层的统一原则,例如信息论或因果推理的角度来理解导航问题?如果从信息获取和不确定性降低的角度重新审视导航任务,可能会得到更通用的解决方案。

\textbf{权世航:}你的想法很有启发性。我开始意识到,优秀的研究不仅在于解决当前问题,更在于为未来研究指明方向。UniGoal在统一表示和零样本泛化方面做出了重要贡献,但也揭示了许多值得进一步探索的问题。

\textbf{康问樵:}完全同意。具身智能是一个充满挑战的领域,需要融合感知、推理、规划和控制等多个方面。UniGoal展示了符号化方法与神经网络方法结合的潜力,这种混合范式可能是未来的发展方向。我们在讨论中提出的许多问题,如不确定性建模、多模态融合、实时性优化等,都是具身智能研究的核心挑战。通过这次深入学习和讨论,我对该领域有了更全面的认识,也激发了对未来研究方向的思考。最重要的是,我们学会了如何批判性地阅读论文,不仅理解其贡献,更要思考其局限性和改进空间,这对科研能力的培养至关重要。

\section{总结与展望}

通过对UniGoal论文的深入学习和讨论,我们系统掌握了通用零样本目标导航的核心技术。论文提出的统一图表示方法、多阶段探索策略和黑名单机制,为解决跨任务导航问题提供了优雅的解决方案。实验结果验证了方法的有效性,在多个基准测试上达到了领先性能。

更重要的是,通过批判性的思考和讨论,我们深刻理解了方法的优势和局限性。统一表示的思想具有普遍意义,但如何在保留信息的同时实现真正的通用性仍是开放问题。零样本泛化能力令人印象深刻,但感知错误的传播、实时性约束和不确定性处理等挑战需要进一步解决。

具身智能是人工智能的前沿领域,连接着虚拟智能与物理世界。导航作为基础能力,其研究进展将直接影响智能机器人的实用化进程。UniGoal展示的符号推理与神经网络结合的范式,可能代表了未来发展方向。随着大语言模型和多模态模型的不断进步,我们期待看到更强大、更通用的具身智能系统出现。

这次大作业不仅让我们学习到了前沿的技术方法,更重要的是培养了科研思维。如何提出有价值的问题,如何批判性地分析现有工作,如何从多个角度思考改进方向,这些能力对未来的学习和研究至关重要。感谢具身智能课程提供了这样的学习机会,让我们能够深入接触学术前沿,培养科研素养。
